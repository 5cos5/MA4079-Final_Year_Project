\chapter{Introduction}

\section{Background}
In the maritime industry, bunker refers to the fuel used on ships. There are several types of bunker available for different applications \cite{noauthor_products_2024}. In 2023, the Port of Singapore was the top bunkering port in the world with over 51 million metric tons \cite{noauthor_cautious_2024} of bunker sold, supplying thousands of vessels which call at the port annually. Bunker is usually delivered to ships by a bunker barge. Typically, a bunker barge collects bunker at the oil terminal before delivering to multiple ships at the anchorage in a trip. As illustrated in Figure \ref{Fig:STS}, bunker in the barge is transferred to customer ships through ship-to-ship transfer. When necessary, the barge returns to the oil terminal to top up its storage to service more customers. This process is known as bunkering.

\begin{figure}
\includegraphics{sts_transfer.jpg}
\caption{Ship to ship transfer of bunker from barge to vessel}
\label{Fig:STS}
\end{figure}

\section{Motivation}
With 41 bunker suppliers \cite{noauthor_list_2024} in the Port of Singapore, bunkering is a highly competitive industry. Hence, each supplier have incentive to find efficient routes for their bunker barges to deliver bunker to customer vessels. Improving route efficiency enhances competitiveness, reduces costs and boosts profitability of the supplier. This problem of finding the shortest routes while servicing the most customers is termed as the bunker supply problem and is the central problem to be studied in the this project. 

Furthermore, to reduce environmental pollution and comply with international regulations, vessels have been using a wide range of bunker. As a result there has been an increase in complexity in bunkering operations with many different types of bunker.

Current methods of routing are typically manual, relying on the experience of staff to determine an efficient route. Therefore, there is a need to automate this process, and find routes that are more efficient than those identified by human operators. 

\section{Objective and Scope}

The aim of this project is to design an automated scheduling and routing system for bunkering operations. The routes generated will maximise revenue of bunker delivered and minimise travel distance of bunker barges. \par 
To ensure that the routes generated are feasible, problem constraints like the capacity of each compartment within the barge and availability of each customer for delivery are considered. Following which, the performance of the system will be evaluated using benchmark datasets to demonstrate its effectiveness and functionality. With this system, the bunker supplier can expect to obtain efficient routes for their bunker barges, within a reasonable time frame.

\section{Organisation of Report}
The report is organised into 6 chapters.The respective focus of each 
chapter is briefly described below. \par
\begin{itemize}
\item Chapter 1 provides an introduction and motivation for the problem.
\item Chapter 2 reviews current literature on routing problems, and introduce methods to tackle those problems. From this review, the author decides to apply linear programming and heuristic methods to solve the problem.
\item Chapter 3 gives the detailed mathematical formulation of the problem, including the
assumptions adopted for the problem, notation of decision variables, problem constraints 
and the objective function of the problem.
\item Chapter 4 describes the heuristic Adaptive Large Neighbourhood search (ALNS) used to solve the problem.
\item Chapter 5 describes the implementation details of the 2 proposed methods and results from testing the effectiveness of the 2 methods. 
\item Chapter 6 discusses future improvements and conclusion of the report.
\end{itemize}

