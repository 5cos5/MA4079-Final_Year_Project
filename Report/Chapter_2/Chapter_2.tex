\chapter{Literature Review}
In this chapter, we will discuss related studies which would help in the development of a routing system.

\section{Vehicle Routing Problem}
The vehicle routing problem (VRP) is a problem first formulated by George Dantizg in 1959 \cite{dantzig_truck_1959} and is a well-researched problem in operations research. The VRP aims to find the shortest route for a set of vehicles to visit a set of customers. 

The VRP can be adapted to fit the bunker supply problem. The objective is modified to maximise profits, defined as the difference between total revenue from bunker delivered and total cost of travel. In the literature, this is known as the profitable tour problem. Another similar problem in literature is the orienteering problem, which aims to maximise revenue, while keeping route distances within a certain limit.

Furthermore, the following additional constraints have to be respected.  
\begin{enumerate}
\item Capacity: Each bunker barge can only carry a limited amount of bunker.
\item Time Windows: Each vessel can only be visited within a specific time window.
\item Multi Trip: Each bunker barge is allowed to perform multiple trips in one planning window.
\item Multi Compartment: In each barge, there are multiple compartments for the different types of bunker.
\item Dynamic Service Time: Service time at the terminal to top up bunker is unknown beforehand as it is dependent on the amount of bunker delivered in the previous trip.
\item Continuous Planning Window: Due to the 24/7 operations, route planning would be done continuously in batches thus, it cannot be assumed that routes planned would start at the terminal.
\end{enumerate}

The VRP is a NP-hard problem \cite{lenstra_complexity_1981}, which underlines the difficulty of finding an optimal route for large problem instances. It is further noted that the addition of time windows constraints makes it such that even finding a feasible route is  NP-hard \cite{savelsbergh_local_1985}.

With these additional set of constraints placed on the VRP, it becomes equivalent to the bunker supply problem. Consequently, we can apply common methods to solve the VRP to solve the bunker supply problem. In existing literature, it is difficult to find papers which discuss exactly the same problem with the same constraints. Thus, modifications would be needed to apply proposed methods to solve the bunker supply problem.

In the VRP, concepts such as vehicle, node/customer and depot are used to describe various elements in the problem. In the bunker supply problem, these would be equivalent to barge, vessel and terminal respectively. These terms would be used interchangeably in this project.

\section{Exhaustive Search Method}
From research done on the bunker supply problem by Wang \cite{wang_optimization_2019}, the bunker supply problem is be solved using an exhaustive search method. The best route is found by iterating through all possible routes to find the best one. However, such a method is of factorial time complexity, which makes finding the optimal route for more than 10 customers impractical. Additional improvements have been made by Ho \cite{ho_development_nodate}, to split the route of the different barges and have made finding an optimal route for up to 20 customers practical.

\section{Linear Programming}
The VRP can be modelled using linear programming (LP) methods. LP methods are flexible and can take into account the various constraints, such as multiple trips \cite{cattaruzza_vehicle_2018} or multiple commodities \cite{gu_vehicle_2024}. The problem can be modelled using math equations, before being solved using state of the art solver engines. 

A standard LP formulation contains 3 parts, the decision variables, objective function and constraints. The decision variables represents decisions that should be made to solve the problem. The variables have to be set such that it fulfils the constraints of the problem. At the same time, the values of the decision variables minimises the objective function. For example, in the VRP, the decision variables binary variables to decide arcs to take between 2 nodes, with the objective function to minimise the cost of travel across arcs chosen. The constraints ensure that the arcs chosen follow a Hamiltonian path and only visit each node once. For the bunker supply problem, the decision variables, objective function and constraints would be modified to fit the problem. Once these are defined, the LP can be solved using the Simplex algorithm to obtain the optimal solution. For LP with integer decision variables, the integer constraint is first relaxed and the resulting LP solved. Subsequently, and branch and bound algorithm and cutting planes can be introduced to force the decision variables to be integer. This is known as the Branch and Cut.

This method is promising as LP methods have been used to solve a variety of optimisation problems beyond the VRP. Thus, advanced solver engines has been created to solve LPs, enabling us to find a solution quickly. We can thus leverage advancements in state-of-art solver engines to find the optimal solution. Notably, from a previous research study done at Piraeus Port in Greece \cite{corman_decision_2015}, a Mixed Integer Linear Program (MILP) model has been developed to find an optimal routing for bunker supply barges. However, the model produced is inefficient and takes over 10,000 seconds to find an optimal route for 10 customers.

\subsection{Branch Cut and Price}
While direct linear programming methods described above are useful, they often struggle to find optimal solutions for large instances. The branch cut and price (BCP) can be used instead. Direct LP models can be decomposed into a master problem (MP). Instead of using arcs as decision variables, entire feasible routes are used as binary decision variables, with the objective function to chose the best route. This is the MP and for the VRP, and is also known as the set covering problem. However, in large problems, there are often too many feasible routes to chose from, thus directly solving the MP impractical. Hence, only a selected subset of routes is chosen, to form the restricted master problem (RMP), which would have its integer requirements relaxed and solved as a pure linear program. This is done as it is easier to solve pure LP problems with a reduced set of decision variables. By solving the RMP, the duals can be obtained, which can be used to calculate the reduced cost of all other variables not in the RMP. By LP theory, variables that have a negative reduced cost would mean that addition of that variable to the RMP would improve the solution. However, there are many variables not in the RMP thus calculation of the reduced cost of each variable would take a long time. Instead, this calculation is converted into another LP problem, known as the pricing problem. Here, the objective is to find the variable with the lowest reduced cost. 

For the VRP, the pricing problem is the resource constraint shortest path problem (RCSP), where we have to find the route which has the minimum reduced cost while respecting some resource constraints like time or capacity. Once routes with a negative reduced cost is found, they are added to the RMP, which is solved again, to generate another pricing problem which is solved again. This cycle continues until the solution of the pricing problem, the minimum reduced cost, is greater than zero. Thus, no addition of routes to the RMP would improve it's solution. Thus, the solution found by the RMP is the optimal solution to the MP. 

As the RMP has it's binary constraints relaxed, a further branch and bound mechanism may be needed to find optimal binary solutions. To further accelerate the process, cutting planes are added to eliminate non-integer solution and reduce branching.

The BCP has been proven to work well to solve various types of VRPs, such as VRP with time windows \cite{lodi_generic_2019} or multi-trip VRPs \cite{hernandez_branch-and-price_2016}. However, additional modification work has to be done to fit the bunker supply problem, mainly to fit the profit maximisation objective function and multiple different fuel types. However, implementation of the BCP is non-trivial. Furthermore, the state of the art implementations include further improvements to the basic BCP described above, further complicating the process of implementing the BCP method to the bunker supply problem. Instead of manually implementing the MP, RMP and Pricing Problem, a Dantzig-Wolfe Decomposition can be done on the original MILP formulation, which can create the MP, RMP and Pricing problems to be solved algorithmically \cite{dantzig_decomposition_1960}. Various projects are ongoing to automate the Dantzig-Wolfe decomposition \cite{hutchison_experiments_2010}, to make the implementation of the BCP easier. However, there are difficulties in detecting the underlying structure of the MILP to split into the MP and pricing problem. As such, current automated methods to convert the MILP to a BCP may not be as efficient as directly solving the MILP using state-of-the-art solver engines. 

\section{Constraint Programming}
Similar to Linear Programming, Constraint Programming (CP) allows us to translate the bunker supply problem into a set of logical constraints, which are then solved by constraint propagation, to find the optimal solution. However, unlike LP where the problem have to be modelled by a set of math equations, CP allows the model to be modelled using math equations and logical constraints. This allows the problem to be potentially modelled in a more efficient manner. CP is first used to solve transportation problems in \cite{puget_object_1993}. More recently, CP has been applied to solve the team orienteering problem, which is similar to the VRP but with a revenue maximisation objective \cite{gedik_constraint_2017}. However, unlike LP, constraints that can be used to model the problem is dependent on the solver used. Also, compared to MILP, the literature is more sparse on CP methods. As such, CP is not considered to solve the bunker supply problem.

\section{Heuristics}
As VRP is NP-hard, exact methods described do not scale well to find optimal solutions for large problems. Instead, we can introduce methods which find approximate solutions in polynomial time. An example are heuristic algorithms which are known to scale better to large problem instances. From a practical perspective, a method which is able to find a good solution in a quick and robust manner are acceptable. Being able to find the optimal solution is usually of less importance.

Examples of heuristic methods are Genetic Algorithm (GA) or Adaptive Large Neighbourhood Search (ALNS) which has been used to solve VRPs. These methods are able to quickly converge onto a good solution in reasonable time. However, as compared to MILP methods, there is no guarantee that the optimal solution would be found. \\

GAs can trace its roots to the evolution process of living organisms as seen in nature. Application of evolution in computing was first proposed by Turing \cite{turing_computing_1950}. In GA, a candidate solution is represented as a chromosome. At the start, a set of chromosomes are generated and evaluated according to a fitness function. Subsequently, a selected number of chromosomes is chosen to undergo recombination. The chromosomes between 2 candidate solutions are mixed to create new candidate solutions also known as offsprings. These offsprings are then evaluated by the fitness function and the cycle repeats. During the creation of offsprings, some mutation may be added to randomly change some parts of the chromosomes. This added randomness can prevent the GA from being stuck in a local minima. \\

ALNS was first created to solve the VRP with pick-ups and deliveries \cite{ropke_adaptive_2006}. ALNS is an extension of an earlier idea known as Large Neighbourhood Search (LNS) \cite{goos_using_1998}. In LNS, the route is repeatedly destroyed and repaired using operators, until a good route is found. This works as it allows the algorithm to search repeatedly search different large neighbourhoods of the search space. allowing it to find a good solution across a large region of the search space. In ALNS, this idea is extended by selecting from a set of destroy and repair operators probabilistically based off their weights. The weight of each operator is adaptively adjusted based off the operator's previous performance. This makes it more likely that well-performing operators are chosen in subsequent iterations. Thus, the search becomes more efficient as good operators are chosen to search for a good solution.

As the name suggest, the ALNS is able to search a large neighbourhood space efficiently, allowing the heuristic to find a good solution. At the same time, due to different problem complexity and structure, it can be difficult to find a single destroy and repair operator that would work well for different problems. ALNS allows us to use a set of operators and selects the best ones to be used. This enhances the flexibility and robustness of ALNS for different problem types, making it an attractive solution method.


\section{Machine Learning}
With advancements in Machine Learning models, it is also possible to solve VRPs using machine learning methods. There are many different types of models under the machine learning umbrella such as attention models. Attention models are extremely powerful and is notably used in Large Language Models (LLM) which can provided generalised reasoning capabilities \cite{vaswani_attention_2023}. The first attention model used in routing problems involves training a pointer network \cite{vinyals_pointer_2017} to solve the Travelling Salesman Problem (TSP), which is a simplified version of the VRP. More recently, attention models \cite{kool_attention_2019} has been used to solve VRP problems. Attention models are used as these models are able to contextualise well between the inputs. This means that it is able to weigh the demands and time windows between the different input customers to determine which should be the next customer to service. \\

In an attention model, the problem information, such as customer location, time windows, demands, current existing route are taken as inputs into the input encoder layer. The encoder layer transforms the problem information into embeddings which would be used by the model.

Next, the embeddings are put through a series of attention layers each with trainable weights. Between each attention layer is an activation function. Once the embeddings are passed through the attention layer, it goes to the output decoder layer, which decodes the embeddings into the customer which should be visited next. This can be added into the route, and reused as the inputs into the attention model. By repeating this process until all customers are served or the end time window is reached, we would be able to obtain a good route.

As obtaining the ground truth is difficult, these models are trained by pitting it against itself. This allows the model to learn what weights are suitable in order to produce a good route. 

These attention models are promising as most of the computational resources are invested beforehand during the training stage. Only a relatively small fraction of computational resources is used during inference, which is where the actual solving of a VRP problem occurs.

However, training and inference of attention models require use of Graphics Processor Unit, (GPU) which can be costly. Also, pre-trained models are usually designed to only take in a certain number of inputs. Thus, there is a limit imposed on the number of customers a model can solve for, when the model is trained.

\section{Quantum Computing}
Quantum Computers (QC) take advantage of the quantum properties of qubits to perform computation. This allows us to take advantage of quantum effects like superposition and entanglement to create speed-ups compared to a classical computer. Being able to out-perform state-of-the-art classical methods is known as the quantum advantage.

QCs are currently noisy intermediate-scale quantum (NISQ) devices \cite{preskill_quantum_2018}. This means that there is a limit to the number of qubits available (100s), and the qubits available are noisy within a limited coherence time. Thus, QCs are prone to errors, and quantum error correction (QEC) techniques are needed to ensure that calculations from QCs are accurate. As such, alogrithms can only employ quantum circuits of limited depths. An example of a suitable algorithm is Quantum Approximation Optimisation Algorithm (QAOA) \cite{farhi_quantum_2014} which have been employed to solve VRPs \cite{azad_solving_2023}. The QAOA is a hybrid algorithm, which solves classically difficult parts of the problem using a quantum computer and the remaining parts solved using a classical computer.

Given the current technological limitations, QC is unable to solve any meaningful size of VRP. Also, for NP-hard combinatorial optimisation problems like the VRP, QC is not expected to solve them in P-time. Despite these limitations, it is of research interest to find quantum methods that can find better approximate solutions quicker or cheaper than classical methods. It is an open problem if a quantum advantage exist for combinatorial optimisation problems \cite{barak_classical_2022}.

\section{Hybrid Methods}
While the above methods have been separated into different types, in reality, they have be combined to form hybrid methods, taking advantage of the relative strengths of each method. For example, heuristics and machine learning \cite{huang_branch_2021} have been used to improve the branch and bound mechanism of LP methods. Similarly, LP \cite{schmid_hybridization_2010} have been included as one of the operators in the heuristic, creating matheuristics methods. Also, in the BCP, the pricing problem can be solved using CP \cite{gualandi_constraint_2013} or heuristics.


\section{Research Gaps}
Due to the NP-hard nature of the problem, the exhaustive search method proposed by \cite{wang_optimization_2019}, \cite{ho_development_nodate} would be infeasible for large problem instances. % Also, as noted by \cite{ho_development_nodate}, more work needs to be done to resolve implementation issues to leverage the computational power of multi-core computers. 
Thus, the author has decided to explore linear programming and heuristic methods to tackle the problem. Due to difficulty, Branch Cut and Price method is not used. Due to resource constraints, the author is unable to implement machine learning methods. The current developments of quantum computers make it impractical to use quantum methods beyond the theory stage. \\

Reviewing literature on linear programming methods, the author is only aware of one research study which fully considers all 6 constraints \cite{corman_decision_2015}. However, it's performance is sub-optimal and the author has identified inefficiencies in the math formulation. Research considering multiple trips \cite{cattaruzza_vehicle_2018} is the foundation for the math model developed. This is used as most of the constraints except for multi-compartments are already modelled. \\

However, limitations of LP methods has led the author to explore heuristic methods. The ALNS heuristic is employed as an alternative to the MILP model. ALNS is chosen as it is designed to solve VRP, and from existing literature, has good performance. This makes ALNS a promising candidate solution to the bunker supply problem. The ALNS implementation in this paper is guided mainly by research done by the original paper \cite{ropke_adaptive_2006} and a modern implementation to solve a similar problem \cite{yu_adaptive_2024}.
