\chapter{Mathematical Formulation}

\section{Problem Description}
In the bunker supply problem, the bunker supplier would have $\mathcal{K}$ number of barges with limited capacity $Q$ for each type of bunker $f$. The barges would have to supply all potential customer vessels $\mathcal{D}$, each having different demands for bunker and bunker type. The barge visits vessels to supply bunker before returning back to the terminal to top-up. This is considered as a trip. A barge is expected to make multiple trips during a planning window. The time required to top-up at the terminal would be linearly related to the amount of fuel required to top-up the barge back to full capacity. The barge can only supply the vessel within the start and end time windows of the vessel. If the barge reaches the vessel early, it is permitted to wait, although this would imply a less efficient route. In most literature and this project, the end time window is the last time possible that the barge can visit the vessel. However, in practical applications, the end time window is usually the last time possible before the barge have to leave the vessel. As such, the end time window may have to be pushed forwards by the service time in order for the solutions generated to be feasible. \par
Routes would be planned on a rolling planning window basis. This means that barge routes would be frequently updated, as new vessels demand top-up of fuel. Hence, at the start of a new planning window, the barge may not be at the terminal and instead be out servicing or travelling, as planned by the previous planning window. As such, the model should allow the operator to input the starting conditions of the various barges. The barges may not necessarily start at the terminal and this would have to be handled appropriately. \par
The problem assumes that the barge would top up to full capacity every time it arrives at the terminal. It is also assumed that vessels are serviced in one visit and cannot be serviced multiple times (no split deliveries allowed). \par

We set up an undirected graph with each vertex $i$ representing a customer vessel. Each edge on the graph is represented by $x_{ij}^{kr}$ and would have a cost to travel represented by $C_{ij}$. A trip $r$ is defined as a route which starts at the terminal, visits various vessels and ends at a terminal. During one planning window, a barge $k$ can undergo $\mathcal{R}$ many trips, as long as it does not breach the time windows.

\section{Parameters}
$T_{ij}$: travel time required to travel from vessel $i$ to vessel $j$\\
$C_{ij}$: cost to travel from vessel $i$ to vessel $j$ equivalent to $\zeta*T_{ij}$ where $\zeta$ is the cost per unit time to operate the vessel\\
$S_{i}$: time required to service vessel $i$\\
$A_{i}$: Start time window of vessel $i$\\
$B_{i}$: End time window of vessel $i$\\
$Q_{f}^{k}$: capacity to carry fuel type $f$ on vessel $k$\\
$N_{if}$: fuel type $f$ required by vessel $i$\\
$R_{if}$: revenue to supply fuel type $f$ to vessel $i$ equivalent to $N_{if} * Price_{f}$\\

\section{Sets}
$\mathcal{D}$: set of vessels $i$ to supply \\
$\mathcal{N}$: set of vessels $i$ to supply \& bunker terminal equivalent to $\mathcal{D} \bigcup {0}$\\
$\mathcal{M}$: set of vessels $i$ to supply \& bunker terminal \& dummy end at bunker terminal equivalent to  $\mathcal{D} \bigcup {0} \bigcup {N+1}$\\
$\mathcal{K}$: set of barges $k$\\
$\mathcal{R}$: set of trips $r$ undertaken by barge $k$\\
$\mathcal{F}$: set of fuel types $f$

\section{Decision Variables}
$x_{ij}^{kr}$: 1 if barge $k$ travels from vessel $i$ to vessel $j$ on trip $r$, 0 otherwise. \\
$y_{i}^{kr}$: 1 if vessel $i$ is visited by barge $k$ on trip $r$, 0 otherwise. \\
$t_{i}^{kr}$: Time when barge $k$ arrives at vessel $i$ on trip $r$. \\
$s_{0}^{kr}$: Service time at the depot at the start of trip $r$ for barge $k$.\\
$l_{if}^{kr}$: Fuel load of type $f$ on barge $k$ on trip $r$ at the start of the trip ($i = 0$) or at the end of the trip ($i = N+1$). \\
$u_{if}^{kr}$: Amount of fuel of type $f$ delivered to vessel $i$ by barge $k$ on trip $r$ 

\section{Objective Function}
Objective function is to maximise the sum of revenue generated from supplying bunker minus the cost of travel between vessels. \\
\begin{equation} \label{eq:obj_fn}
maximise \sum_{i \in \mathcal{D}}\sum_{k \in \mathcal{K}}\sum_{r \in \mathcal{R}}\sum_{f \in \mathcal{F}} y_{i}^{kr} *R_{if} - 
\sum_{i \in \mathcal{N}}\sum_{j \in \mathcal{N}}\sum_{k \in \mathcal{K}}\sum_{r \in \mathcal{R}} x_{ij}^{kr} * C_{ij}
\end{equation}


\section{Constraints}

\subsection{Route Constraints}

\begin{equation} \label{eq:2}
\sum_{k \in \mathcal{K}} \sum_{r \in \mathcal{R}} y_{i}^{kr} \leq 1 \quad \forall i \in \mathcal{D}
\end{equation}
\begin{equation} \label{eq:3}
\sum_{j \in \mathcal{N}} x_{ij}^{kr} = \sum_{j \in \mathcal{N}} x_{ji}^{kr} = y_{i}^{kr} 
\quad \forall i \in \mathcal{N}, \, k \in \mathcal{K}, \, r \in \mathcal{R}
\end{equation}

Equation \ref{eq:2} ensures that each vessel is visited at most once.

Equation \ref{eq:3} are arc flow constraints to ensure the route goes in a loop starting and ending at the depot.

\subsubsection{Sub-tour elimination}
Sub-tours are routes which are cyclic, but do not start and end at the terminal. This is a problem as all barges are should start and end their routes at the terminal. To eliminate this, various sub-tour elimination constraints (SEC) has been proposed to remove sub-tours. The following is a non-exhaustive list of different methods to remove sub-tours.
\begin{enumerate}
\item Danzig–Fulkerson–Johnson (DFJ): This method iteratively adds constraints to connect the sub-tours which exist in the solution, and solve the problem with the added constraints \cite{dantzig_solution_1954}. 
\item Miller–Tucker–Zemlin (MTZ): This method adds an ordering continuous decision variable at each node, to order the visits. Thus, routes can be forced to start at the terminal and consequently, to form a cycle and end at the terminal \cite{miller_integer_1960}.
\item  Gavish–Graves (GG): This method adds a flow continuous decision variable at each arc. Constraints are added such that all flows must eventually reach back to the terminal. Thus, all routes have to end at the terminal \cite{gavish_travelling_1978}.
\end{enumerate}

In this formulation, MTZ method is a natural extension of the problem. As time windows of each customer have to be respected by the servicing barge, a decision variable to track visit time $t_{i}^{kr}$ has to be added. Consequently, this variable also acts as the ordering variable in the MTZ method.

\subsection{Time Constraints}

\begin{equation} \label{eq:72}
t_{0}^{kr} + s_{0}^{kr} + T_{0j} \leq t_{j}^{kr} + M_{0j}(1 - x_{0j}^{kr}) 
\quad \forall j \in \mathcal{D}, \, k \in \mathcal{K}, \, r \in \mathcal{R}
\end{equation} %checekd
\begin{equation} \label{eq:71}
t_{i}^{kr} + S_{i} + T_{ij} \leq t_{j}^{kr} + M_{ij}(1 - x_{ij}^{kr}) 
\quad \forall i,j \in \mathcal{D}, \, k \in \mathcal{K}, \, r \in \mathcal{R},\, i \ne j
\end{equation} %checked
\begin{equation} \label{eq:66}
t_{i}^{kr} + S_{i} + T_{i0} \leq t_{N+1}^{kr} + M_{i0}(1 - x_{i0}^{kr}) 
\quad \forall i \in \mathcal{D}, \, k \in \mathcal{K}, \, r \in \mathcal{R}
\end{equation} %checked
\begin{equation} \label{eq:67}
t_{N+1}^{kr} \leq t_0^{k,r+1} 
\quad \forall k \in \mathcal{K}, \, r \in \mathcal{R} \setminus \{N-1\}
\end{equation} %checked
\begin{equation} \label{eq:69}
t_{N+1}^{kr} \leq T_H 
\quad \forall k \in \mathcal{K}, \, r \in \mathcal{R}
\end{equation} %checked
\begin{equation} \label{eq:68}
A_{i}  \leq t_i^{kr} \leq B_{i }
\quad \forall i \in \mathcal{D}, \, k \in \mathcal{K}, \, r \in \mathcal{R}
\end{equation} %checked
\begin{equation} \label{eq:73}
s_{0}^{kr} = \beta (FC - \sum_{f \in \mathcal{F}}(l_{N+1,f}^{k,r-1}) 
\quad \forall k \in \mathcal{K}, \, r \in \mathcal{R} \setminus \{0\}
\end{equation}
\begin{equation} \label{eq:74}
s_{0}^{k0} = 0 \quad \forall k \in \mathcal{K}
\end{equation}

where $FC$ is the full capacity of the barge

where $\beta$ is the rate of fuel transfer at the depot

where $M_{ij} = \begin{cases}
B_{0} + FC*\beta + T_{0j} - A_{j} & \forall j \in \mathcal{N}, \\
B_{i} + S_{i} + T_{ij} - A_{j} & \forall i\in \mathcal{D}, \, j \in \mathcal{N} \\
\end{cases} $ 

To ensure that the constraints are linear, the Big M formulation is used. The M coefficient is tightened to as small as possible to make the formulation more efficient.  \\

Equation \ref{eq:72} ensures that the time start at the depot $t_{0}^{kr}$ at the start of the trip is before the time the barge arrives at the first vessel $i$ including servicing and travelling time. This is separate to account for dynamic service time $s_{0}$ at the start of the trip.

Equation \ref{eq:71} ensures that ensures that the time visited at the next vessel $j$ is after the time the barge arrives at the previous vessel $i$ including servicing and travelling time. It allows for the barge to wait at vessel $j$ before arriving.

Equation \ref{eq:66} ensures that the time reached the depot $t_{N+1}^{kr}$ at the end of the trip is after the time the barge arrives at the last vessel $i$ including servicing and travelling time.

Equation \ref{eq:67} ensures that the time start of the next trip $r+1$ at the depot is after the barge has arrived at the depot at the end of the previous trip $r$.

Equation \ref{eq:69} ensures that the end of all trips is before the end time for the planning window.

Equation \ref{eq:68} ensures that start $A_i$ and end $B_i$ time windows of the vessel are respected.

Equation \ref{eq:73} ensures that service time at the start of the trip $s_{0}^{kr}$ is the difference between the full capacity of the barge and the amount of fuel left from the previous trip multiplied by the rate fuel can be transferred onto the barge.

Equation \ref{eq:74} ensures that service time at the for the depot for the first trip is 0. It is assumed that there would be no loading of fuel at the start.

\subsubsection{Arcs Filtering}
\begin{equation} \label{eq:23}
A_{i} + S_{i} +T_{ij} > B_{j}
\end{equation}
If equation \ref{eq:23} is satisfied, the arc $x_{ij}^{kr}$ is infeasible to travel as it would not satisfy time window constraints. Thus, $x_{ij}^{kr}$ is removed from the formulation for all $\mathcal{K}, \mathcal{R}$. This reduces the number of decision variables, making the formulation more efficient. This is especially useful for problems with tight time windows, which would result in the removal of many arcs.
 

\subsection{Load Constraints}

\begin{equation} \label{eq:10}
l_{0f}^{kr} = FC_{f} \quad \forall k \in \mathcal{K}, \, r \in \mathcal{R} \setminus \{0\}, \, f \in \mathcal{F}
\end{equation}
\begin{equation} \label{eq:11}
u_{if}^{kr} = y_{i}^{kr} * N_{if} \quad \forall i \in \mathcal{N},\, k \in \mathcal{K}, \, r \in \mathcal{R}, \, f \in \mathcal{F}
\end{equation}
\begin{equation} \label{eq:12}
l_{N+1,f}^{kr} = l_{0f}^{kr} - \sum_{i \in \mathcal{D}}u_{if}^{kr}   \quad \forall i \in \mathcal{N},\, k \in \mathcal{K}, \, r \in \mathcal{R}, \, f \in \mathcal{F}
\end{equation}

Equation \ref{eq:10} ensures that at the start of the trip, the barge is at full capacity. Except for the first trip where the load would be dependent on the initial conditions. \\

Equation \ref{eq:11} ensures that the fuel delivered to the vessel is equal to the demand if the vessel is visited by the barge. Otherwise, fuel delivered to the vessel is 0 as it is not visited by the barge.\\

Equation \ref{eq:12} ensures that at the end of the trip, the load left in the barge is the difference between the starting initial load and the sum of all fuel delivered to vessels in that trip.


\section{Initial Conditions}
Due to the rolling planning windows, there is a need to define the initial conditions of the various barges. \par
If the plan is at the start of the planning window, there would be no route constraints, start time of all barges would be at $t=0$ and all barges would be at the full load.\par
If the plan is planning for subsequent planning windows, the barge would be starting at the vessel which it is at, based off the previous planning window. Similarly, the time and load constraints would be the time and load of the barge currently.\par
However, due to flow conservation in the model, all routes planned would have to be in a loop. It is not possible to create a route which does not end at the same place where it started. To get around this, the author proposes to add a dummy visit from the depot to the starting location. This would allow for the model to create a loop for a route that starts and end at the depot. Consequently, starting time of the barge have to be adjusted to account for the dummy travelling time from the depot to the starting customer.
\subsection{Route Constraints}
\begin{equation} \label{eq:20}
x_{0,SV^{k}}^{k0} = 1 \quad \forall k \in \mathcal{K}
\end{equation}
where $SV^{k}$ is the starting vessel of each vehicle $k$
\subsection{Time Constraints}
\begin{equation} \label{eq:21}
t_{0}^{k0} = ST^{k} \quad \forall k \in \mathcal{K}
\end{equation}
where $ST^{k}$ is the starting time of each vehicle $k$
\subsection{Load Constraints}
\begin{equation}\label{eq:22}
l_{0f}^{k0} = SF_{f}^{k} \quad \forall k \in \mathcal{K},\, f \in \mathcal{F}
\end{equation}
where $SF_{f}^{k}$ is the starting fuel of each vehicle $k$ for fuel type $f$




